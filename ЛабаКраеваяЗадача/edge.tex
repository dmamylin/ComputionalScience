\documentclass[12pt,a4paper]{article}
\usepackage[utf8]{inputenc}
\usepackage[russian]{babel}
\usepackage{listings}
\usepackage{amsmath}
\usepackage{geometry}
\usepackage{graphicx}
 \geometry{
 a4paper,
 total={210mm,297mm},
 left=5mm,
 right=5mm,
 top=5mm,
 bottom=5mm,
 }

%\topmargin=-1.5cm
%\parindent=24pt
%\parskip=0pt
%\flushbottom
%\lstset{language=python}

\title{Расчетно-графическая работа №3}
\author{Мамылин Дмитрий, МТ-301}

\begin{document}

\DeclareGraphicsExtensions{.png}

\maketitle
\thispagestyle{empty}

\newpage
\thispagestyle{empty}

    \section*{Постановка задачи}
        Дана краевая задача на отрезке $[0, 1]$:\\
        \[
        (1):
        \begin{cases}
        	y'' = y + 16.8 + 7.4 \cdot x(1 - x)\\
	    	y(0) = 0\\
     		y'(1) + y(1) = 2e + 1.7
     	\end{cases}
     	\]\\
     	где $f(x, y, y') = y + 16.8 + 7.4 \cdot x(1 - x)$.\\\\
     	
     	\noindent
        Необходимо решить ее методом стрельбы и методом прогонки, используя:\\
        \vspace{-5mm}
        \begin{itemize}
        \setlength\itemsep{-1.5em}
	        \item Метод Рунге-Кутта четвертого порядка для решения задачи Коши в методе стрельбы\\
    	    \item Метод Ньютона для решения нелинейного уравнения в методе стрельбы\\
        	\item Метод введения фиктивного узла для аппроксимации краевых условий в методе прогонки.
        \end{itemize}
        
    \section*{Точное решение}
        Найдем точное решение задачи $(1)$.\\
        Соответствующее однородное уравнение:\\
        $y'' - y = 0$\\
        Его характеристическое уравнение:\\
        $\lambda^{2} = 1 \Rightarrow \lambda_{1} = -1, \lambda_{2} = 1$\\
        Общее решение однородного: $y(x) = C_{1}e^{-x} + C_{2}e^{x}$\\
        Частное решение неоднородного: $y(x) = 7.4x^{2} - 7.4x - 2$\\
        Тогда общее решение неоднородного: $y(x) = C_{1}e^{-x} + C_{2}e^{x} +
        7.4x^{2} - 7.4x - 2$\\\\
        
        \noindent
        Найдем константы:\\
        $\begin{cases}
	        C_{1} + C_{2} = 2\\
    	    -C_{1}e^{-1} + C_{2}e + 7.4 = -C_{1}e^{-1} - C_{2}e + 2 + 2e + 1.7
        \end{cases}$\\
        
        \vspace{2mm}
        \noindent
        Откуда:\\
        $C_{2} = \frac{2e - 3.7}{2e}$\\
        $C_{1} = 2 - C_{2} = \frac{2e + 3.7}{2e}$\\\\
        
        \noindent
        Точное решение:\\
        $y(x) = \frac{2e + 3.7}{2e}e^{-x} + \frac{2e - 3.7}{2e}e^{x} 
        + 7.4x^{2} - 7.4x - 2$\\
        \includegraphics[scale=0.5]{exact_solution_graph}
    
    \newpage
    \section*{Метод стрельбы}
        Рассматриваем вспомогательную задачу Коши с некоторым параметром $\mu$:\\
        \[
        (2):
        \begin{cases}
        	y'' = y + 16.8 + 7.4 \cdot x(1 - x)\\
	        y(0) = 0\\
    	    y'(0) = \mu\\
        \end{cases}
        \]
        Сделаем замену: $z = y'$, тогда система $(2)$ примет вид:\\
        \[
        (2'):
        \begin{cases}
        	y' = z\\
        	z' = y + 16.8 + 7.4 \cdot x(1 - x)\\
        	y(0) = 0\\
        	z(0) = \mu\\
        \end{cases}
        \]
        Где:\\
        $f_{1}(x, y, z) = z$\\
        $f_{2}(x, y, z) = f(x, y, z) = f(x, y, y') = y + 16.8 + 7.4 \cdot x(1 - x)$.\\\\
        
        \noindent
        Общий вид метода Рунге-Кутта четвертого порядка для системы ОДУ второго порядка:\\
        $\begin{cases}
        	y' = f_{1}(x, y, z)\\
        	z' = f_{2}(x, y, z)\\
        	y(x_{0}) = y_{0}\\
        	z(x_{0}) = z_{0}
        \end{cases}$\\\\
        
        \noindent
        $y_{i+1} = y_{i} + \frac{1}{6}(k_{i}^{1} +2k_{i}^{2} + 2k_{i}^3 + k_{i}^4)$\\
        $z_{i+1} = z_{i} + \frac{1}{6}(l_{i}^{1} +2l_{i}^{2} + 2l_{i}^3 + l_{i}^4)$\\
        
        \noindent
        Где:\\
        $k_{i}^{1} = hf_{1}(x_{i}, y_{i}, z_{i})$\\
        $l_{i}^{1} = hf_{2}(x_{i}, y_{i}, z_{i})$\\
        $k_{i}^{2} = hf_{1}(x_{i} + \frac{h}{2}, y_{i} + \frac{k_{i}^{1}}{2},
            z_{i} + \frac{l_{i}^{1}}{2})$\\
        $l_{i}^{2} = hf_{2}(x_{i} + \frac{h}{2}, y_{i} + \frac{k_{i}^{1}}{2},
            z_{i} + \frac{l_{i}^{1}}{2})$\\
        $k_{i}^{3} = hf_{1}(x_{i} + \frac{h}{2}, y_{i} + \frac{k_{i}^{2}}{2},
            z_{i} + \frac{l_{i}^{2}}{2})$\\
        $l_{i}^{3} = hf_{2}(x_{i} + \frac{h}{2}, y_{i} + \frac{k_{i}^{2}}{2},
            z_{i} + \frac{l_{i}^{2}}{2})$\\
        $k_{i}^{4} = hf_{1}(x_{i+1}, y_{i} + k_{i}^{3}, z_{i} + l_{i}^{3})$\\
        $l_{i}^{4} = hf_{2}(x_{i+1}, y_{i} + k_{i}^{3}, z_{i} + l_{i}^{3})$\\
        $i = 0, 1, \dotsc, n$\\
        $\{x\}_{i=0}^n$ - сетка на отрезке $[0, 1]$\\
        
        \vspace{2mm}
        \noindent
        Применяем к задаче $(2')$:\\
        $y_{i+1} = y_{i} + \frac{1}{6}(k_{i}^{1} +2k_{i}^{2} + 2k_{i}^3 + k_{i}^4)$\\
        $z_{i+1} = z_{i} + \frac{1}{6}(l_{i}^{1} +2l_{i}^{2} + 2l_{i}^3 + l_{i}^4)$\\
        Где:\\
        $k_{i}^{1} = hz_{i}$\\
        $l_{i}^{1} = h(y_{i} + 7.4x_{i}(1 - x_{i}) + 16.8)$\\
        $k_{i}^{2} = h(z_{i} + \frac{l_{i}^{1}}{2})$\\
        $l_{i}^{2} = h(y_{i} + \frac{k_{i}^{1}}{2} + 7.4(x_{i} + \frac{h}{2})
            (1 - \frac{h}{2} - x_{i}) + 16.8)$\\
        $k_{i}^{3} = h(z_{i} + \frac{l_{i}^{2}}{2})$\\
        $l_{i}^{3} = h(y_{i} + \frac{k_{i}^{2}}{2} + 7.4(x_{i} + \frac{h}{2})
            (1 - \frac{h}{2} - x_{i}) + 16.8)$\\
        $k_{i}^{4} = h(z_{i} + l_{i}^{3})$\\
        $l_{i}^{4} = h(y_{i} + k_{i}^{3} + 7.4x_{i+1}(1 - x_{i+1}) + 16.8)$\\
        
        \noindent
        После этого шага получим набор значений:\\
        $0 = y(0) = y_{0}, y_{1}, y_{2}, \dotsc, y_{n} = y(\mu, 1)$\\
        $\mu = z(0) = z_{0}, z_{1}, z_{2}, \dotsc, z_{n} = z(\mu, 1)$\\\\
        
        \noindent
        Ищем параметр $\mu$. Решаем уравнение $F(\mu) = 0$\\
        где $F(\mu) = y(\mu, 1) + z(\mu, 1) - 2e - 1.7$\\
        Применяем метод Ньютона:\\
        $\mu_{j+1} = \mu_{j} - F(\mu_{j}) / F'(\mu_{j})$, где $j = 0, 1, \dots$\\
        Считаем, что изначально был задан $\mu_{0}$ - некоторый начальный параметр.\\\\
        
		\noindent
		Для поиска $F'(\mu_{j}) = y'_{\mu}(\mu, 1) + z'_{\mu}(\mu, 1)$
		продифференцируем задачу $(2')$ по параметру $\mu$:\\
		\[
		(3):
		\begin{cases}
			y''_{x\mu} = z'_{\mu}\\
			z''_{x\mu} = y'_{\mu}\\
			y'_{\mu}(0) = 0\\
			z'_{\mu}(0) = 1
		\end{cases}
		\]
		Обозначим: $v = y'_{\mu}; u = z'_{\mu}$\\
		Получим новую задачу Коши:\\
		\[
		(3'):
		\begin{cases}
			v' = u\\
			u' = v\\
			v(0) = 0\\
			u(0) = 1
		\end{cases}
		\]
		Где:\\
		$g_{1}(x, v, u) = u$\\
		$g_{2}(x, v, u) = v$\\\\
		
		\noindent
		Для решения задачи $(3')$ применяем метод Рунге-Кутта четвертого порядка:\\
		$v_{i+1} = v_{i} + \frac{1}{6}(s_{i}^{1} +2s_{i}^{2} + 2s_{i}^3 + s_{i}^4)$\\
        $u_{i+1} = u_{i} + \frac{1}{6}(t_{i}^{1} +2t_{i}^{2} + 2t_{i}^3 + t_{i}^4)$\\
        Где:\\
        $s_{i}^{1} = hu_{i}$\\
        $t_{i}^{1} = hv_{i}$\\
        $s_{i}^{2} = h(u_{i} + \frac{s_{i}^{1}}{2})$\\
        $t_{i}^{2} = h(v_{i} + \frac{t_{i}^{1}}{2})$\\
        $s_{i}^{3} = h(u_{i} + \frac{s_{i}^{2}}{2})$\\
        $t_{i}^{3} = h(v_{i} + \frac{t_{i}^{2}}{2})$\\
        $s_{i}^{4} = h(u_{i} + s_{i}^{3})$\\
        $t_{i}^{4} = h(v_{i} + t_{i}^{3})$\\\\
        
        \noindent
        Когда для некоторой заданной точности $\varepsilon > 0$ будет выполнено:
        $|\mu_{j} - \mu_{j+1}| < \varepsilon$, завершаем алгоритм.
        
    \newpage
	\section*{Метод прогонки}
		Применим формулы численного дифференцирования по трем узлам:\\
		$y'_{i} = \frac{y_{i+1} - y_{i-1}}{2h}$\\
		$y''_{i} = \frac{y_{i-1} - 2y_{i} + y_{i+1}}{h^{2}}$\\
		К задаче $(1)$
		(обе формулы имеют порядок точности $O(h^{2})$)\\\\
		
		\noindent
		Проведем преобразования задачи $(1)$:\\
		\[
		\begin{cases}
			y'' = y + 16.8 + 7.4x(1-x)\\
			y(0) = 0\\
			y'(1) + y(1) = 2e + 1.7
		\end{cases}\\
		\]\\
		Подставляем формулы численного дифференцирования:\\
		
		\[
		\begin{cases}
			\frac{y_{i-1} - 2y_{i} + y_{i+1}}{h^{2}} = y_{i} + 16.8 + 7.4x_{i}(1 - x_{i}), i = 1, 2, \dotsc, n-1\\
			\frac{y_{n-1} - 2y_{n} + y_{n+1}}{h^{2}} = y_{n} + 16.8 + 7.4x_{n}(1 - x_{n})\\
			y_{0} = 0\\
			\frac{y_{n+1} - y_{n-1}}{2h} + y_{n} = 2e + 1.7
		\end{cases}
		\]\\
		\hspace{-6mm}
		\[
		\iff
		\]\\
		\hspace{-9mm}
		\[
		\begin{cases}
			(I):     y_{i-1} + (-2 - h^{2})y_{i} + y_{i+1} = h^{2}(16.8 + 7.4x_{i}(1 - x_{i})), i = 1, 2, \dotsc, n-1\\
			(II):   y_{n-1} + (-2 - h^{2})y_{n} + y_{n+1} = h^{2}16.8\\
			(III): y_{0} = 0\\
			(IV):   -y_{n-1} + 2hy_{n} + y_{n+1} = 2h(2e + 1.7)
		\end{cases}
		\]\\
		
		Сложим уравнение $(II)$ с $(IV)$, умножив уравнение $(IV)$ на $-1$:\\
		\[
		\begin{cases}
			y_{0} = 0\\
			y_{i-1} + (-2 - h^{2})y_{i} + y_{i+1} = h^{2}(16.8 + 7.4x_{i}(1 - x_{i})), i = 1, 2, \dotsc, n-1\\
			2y_{n-1} + (-2 - h^{2} - 2h)y_{n} = h^{2}16.8 - 2h(2e + 1.7)
		\end{cases}
		\]\\
		Получим систему уравнений:\\
		\[ \left( \begin{array}{ccccccc}
		1 & 0 & 0 & 0 & 0 & \dots & 0 \\
		1 & (-2-h^{2}) & 1 & 0 & 0 & \dots & 0 \\
		0 & 1 & (-2-h^{2}) & 1 & 0 & \dots & 0 \\
		\dots & \dots & \dots & \dots & \dots & \dots & \dots \\
		0 & 0 & 0 & \dots & 0 & 2 & -2-h^{2}-2h \end{array} \right)
		\cdot
		\left( \begin{array}{c}
		y_{0} \\
		y_{1} \\
		y_{2} \\
		\dots \\
		y_{n} \end{array} \right)
		=
		\left( \begin{array}{c}
		0 \\
		h^2(16.8 + 7.4x_{1}(1 - x_{1})) \\
		h^2(16.8 + 7.4x_{2}(1 - x_{2})) \\
		\dots \\
		h^{2}16.8 - 2h(2e + 1.7) \end{array} \right)
		\]\\
		Заметим:\\
		$|-2 - h^{2}| = 2 + h^{2} > 2$\\
		$|-2 - h^2 - 2h| = 2 + h^{2} + 2h > 2$\\
		То есть выполнено диагональное преобладание, значит, метод прогонки применим.
		
	\newpage
	\section*{Вывод}
		Метод стрельбы сходится быстрее метода прогонки, поскольку в нем используется
		метод Рунге-Кутта второго порядка, в то время как в методе прогонки применяются
		методы численного дифференцирования лишь второго порядка.
\end{document}
