\documentclass[12pt,a4paper]{article}
\usepackage[utf8]{inputenc}
\usepackage[russian]{babel}
\usepackage{listings}
\usepackage{geometry}
 \geometry{
 a4paper,
 total={210mm,297mm},
 left=7mm,
 right=7mm,
 top=8mm,
 bottom=8mm,
 }

\topmargin=-1.5cm
\parindent=24pt
\parskip=0pt
\flushbottom
\lstset{language=C}

\title{Расчетно-графическая работа №1}
\author{Мамылин Дмитрий, МТ-301}

\begin{document}

\maketitle
\thispagestyle{empty}

\newpage
\thispagestyle{empty}

    \section*{Постановка задачи}
        Дан интеграл $\int_2^3 \! e^{-\cos(x)} \, \mathrm{d}x$.
        Необходимо вычислить значение интеграла по двум составным формулам:
        по формуле средних прямоугольников и формуле Грегори с шагом
        $0.1, 0.05, 0.025$, оценить погрешность по Рунге и вычислить
        значение, используя формуле Гаусса (по двум узлам). \\
        
    \section*{Формула средних прямоугольников}
        \begin{itemize}
            \item Шаг $h = 0.1$:\\
                $n = \frac{3 - 2}{0.1} = 10 $;
                $x_{i} = 2 + i \cdot h = 2 + i \cdot 0.1,$ где $i = 0, \cdots, 10$\\
                $\int_2^3 \! e^{-\cos(x)} \, \mathrm{d}x \approx I_{h} = \sum_{k=0}^{9} (3 - 2)
                \cdot (x_{i+1} - x_{i}) \cdot e^{\frac{x_{i+1} + x_{i}}{2}} =
                \sum_{k=0}^{9} (x_{i+1} - x_{i}) \cdot e^{\frac{x_{i+1} + x_{i}}{2}}
                \approx 2.18713$.
                
            \item Шаг $h = 0.05$:\\
                $n = \frac{3 - 2}{0.05} = 20 $;
                $x_{i} = 2 + i \cdot h = 2 + i \cdot 0.05,$ где $i = 0, \cdots, 20$\\
                $\int_2^3 \! e^{-\cos(x)} \, \mathrm{d}x \approx I_{\frac{h}{2}} =
                \sum_{k=0}^{19} (3 - 2) \cdot (x_{i+1} - x_{i}) \cdot e^{\frac{x_{i+1} + x_{i}}{2}} =
                \sum_{k=0}^{19} (x_{i+1} - x_{i}) \cdot e^{\frac{x_{i+1} + x_{i}}{2}}
                \approx 2.18681$.
                
            \item Шаг $h = 0.025$:\\
                $n = \frac{3 - 2}{0.05} = 40 $;
                $x_{i} = 2 + i \cdot h = 2 + i \cdot 0.025,$ где $i = 0, \cdots, 40$\\
                $\int_2^3 \! e^{-\cos(x)} \, \mathrm{d}x \approx I_{\frac{h}{4}} =
                \sum_{k=0}^{39} (3 - 2) \cdot (x_{i+1} - x_{i}) \cdot e^{\frac{x_{i+1} + x_{i}}{2}} =
                \sum_{k=0}^{39} (x_{i+1} - x_{i}) \cdot e^{\frac{x_{i+1} + x_{i}}{2}}
                \approx 2.18674$.
                
            \item Погрешность:\\
                $R_{h} \sim 2 = m;$ применим метод Рунге для оценки погрешностей:\\
                $R_{\frac{h}{2}} \approx \frac{I_{\frac{h}{2}} - I_{h}}{2^{m} - 1} =
                \frac{2.18681 - 2.18713}{2^2 - 1} = \frac{-0.00032}{3} \approx -1.1 \cdot 10^{-4}$\\
                $R_{\frac{h}{4}} \approx \frac{I_{\frac{h}{4}} - I_{\frac{h}{2}}}{2^{m} - 1} =
                \frac{2.18674 - 2.18681}{2^2 - 1} = \frac{-0.00032}{3} \approx -2.3 \cdot 10^{-5}$
        \end{itemize}
        
    \section*{Формула Грегори}
        \begin{itemize}
            \item Шаг $h = 0.1$:\\
                $\int_2^3 \! e^{-\cos(x)} \, \mathrm{d}x \approx I_{h} = 
                \frac{5 \cdot h}{12} \cdot (e^{-\cos(2)} + e^{-\cos(3)}) + h \cdot
                \sum_{k=1}^{9} e^{-\cos(x_{k})} + \frac{h}{12} \cdot
                (e^{-\cos(x_{1})} + e^{-\cos(x_{9})}) \approx 2.18663$.
                
            \item Шаг $h = 0.05$:\\
                $\int_2^3 \! e^{-\cos(x)} \, \mathrm{d}x \approx I_{h} = 
                \frac{5 \cdot h}{12} \cdot (e^{-\cos(2)} + e^{-\cos(3)}) + h \cdot
                \sum_{k=1}^{19} e^{-\cos(x_{k})} + \frac{h}{12} \cdot
                (e^{-\cos(x_{1})} + e^{-\cos(x_{19})}) \approx 2.18670$.
                
            \item Шаг $h = 0.025$:\\
                $\int_2^3 \! e^{-\cos(x)} \, \mathrm{d}x \approx I_{h} = 
                \frac{5 \cdot h}{12} \cdot (e^{-\cos(2)} + e^{-\cos(3)}) + h \cdot
                \sum_{k=1}^{39} e^{-\cos(x_{k})} + \frac{h}{12} \cdot
                (e^{-\cos(x_{1})} + e^{-\cos(x_{39})}) \approx 2.18671$.
                
            \item Погрешность:\\
                $R_{h} \sim 3 = m;$ применим метод Рунге для оценки погрешностей:\\
                $R_{\frac{h}{2}} \approx \frac{I_{\frac{h}{2}} - I_{h}}{2^{m} - 1} =
                \frac{2.18670 - 2.18663}{2^3 - 1} = \frac{0.00007}{7} \approx 1 \cdot 10^{-4}$\\
                $R_{\frac{h}{4}} \approx \frac{I_{\frac{h}{4}} - I_{\frac{h}{2}}}{2^{m} - 1} =
                \frac{2.18671 - 2.18670}{2^3 - 1} = \frac{0.00001}{7} \approx 1.4 \cdot 10^{-6}$
        \end{itemize}
        
    \section*{Формула Гаусса}
        Используем оптимальные узлы и веса на отрезке $[-1, 1]: x_{1, 2} = \pm \sqrt{\frac{1}{3}},
        w_{1, 2} = 1$.\\
        Сведем исходный интеграл по промежутку $[2, 3]$ к интегралу по промежутку $[-1, 1]$:\\
        $\int_2^3 \! e^{-\cos(x)} \, \mathrm{d}x = \frac{1}{2} \cdot
        \int_{-1}^{1} \! e^{-\cos(0.5 \cdot z + 2.5)} \, \mathrm{d}z \approx
        0.5 \cdot (e^{-\cos(0.5 \cdot \sqrt{\frac{1}{3}} + 2.5)} + e^{-\cos(0.5 \cdot
        (-\sqrt{\frac{1}{3}}) + 2.5)}) \approx 2.186797$
    
\newpage    
    
    \section*{Вывод}
        Метод Рунге позволяет оценивать погрешности по ходу вычислений, причем для 
        методов большего порядка точность будет выше, чем для методов меньшего.
    
    \section*{Приложение}
        \begin{lstlisting}
#include <stdio.h>
#include <math.h>

const double A = 2.0;
const double B = 3.0;
const double STEPS[] = { 0.1, 0.05, 0.025 };

double f(double x) {
    return exp(-cos(x));
}

double midRectangles(double a, double b, double h) {
    double result = 0.0;
    int i;
    int n = (int)floor((b - a) / h);

    for (i = 0; i < n; i++) {
        double xPrev = a + i * h;
        double xNext = a + (i + 1) * h;
        double middlePoint = (xPrev + xNext) / 2.0;

        result += f(middlePoint);
    }

    return result * h;
}

double gregory(double a, double b, double h) {
    double x1 = a + h;
    double xn_1 = b - h;
    double result = 0.0;
    int i;
    int n = (int)floor((b - a) / h);

    result += (5 * (f(a) + f(b)) + f(x1) + f(xn_1)) / 12.0;
    
    for (i = 1; i < n; i++) {
        result += f(a + i * h);
    }

    return result * h;
}
        \end{lstlisting}
        
\end{document}
