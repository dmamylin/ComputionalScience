\documentclass[12pt,a4paper]{article}
\usepackage[utf8]{inputenc}
\usepackage[russian]{babel}
\usepackage{listings}
\usepackage{geometry}
\usepackage{graphicx}
 \geometry{
 a4paper,
 total={210mm,297mm},
 left=5mm,
 right=5mm,
 top=5mm,
 bottom=5mm,
 }

%\topmargin=-1.5cm
%\parindent=24pt
%\parskip=0pt
%\flushbottom
\lstset{language=python}

\title{Расчетно-графическая работа №2}
\author{Мамылин Дмитрий, МТ-301}

\begin{document}

\DeclareGraphicsExtensions{.png}

\maketitle
\thispagestyle{empty}

\newpage
\thispagestyle{empty}

    \section*{Постановка задачи}
        Дана задача Коши:\\*
        \begin{center}
            $y' = \frac{3}{2}\cdot\sqrt{x}$\\
            $y(0) = -1$
        \end{center}\smallskip
        Будем решать задачу на отрезке $[0, 1]$. Зафиксируем на этом отрезке
        равномерную сетку $\{x_{i}\}_{i=0}^{n}$ с шагом $h$, где $x_{0} = 0, x_{1} = 1$.\\
        Задачу необходимо решить явным методом Эйлера, неявным методом Эйлера и
        явным двушаговым методом Адамса.\\
        Точное решение: $y(x) = x^{\frac{3}{2}} - 1$\\
        \includegraphics[scale=0.5]{exactSolutionGraph}
        
    \section*{Решение явным методом Эйлера}
    	Общий вид: $y_{i+1} = y_{i} + h \cdot f(x_{i}, y_{i})$, где $h$ - шаг.\\
        Формула: $y_{i+1} = y_{i} + h \cdot \frac{3}{2}\cdot\sqrt{x_{i}}$.\\*

		\noindent
        \includegraphics[scale=0.45]{explicitEulerGraph_Step=0_5}
        \includegraphics[scale=0.45]{explicitEulerGraph_Step=0_25}\\
        \includegraphics[scale=0.45]{explicitEulerGraph_Step=0_125}
        \includegraphics[scale=0.45]{explicitEulerGraph_Step=0_05}
        
    \section*{Решение неявным методом Эйлера}
    	Общий вид: $y_{i+1} = y_{i} + h \cdot f(x_{i+1}, y_{i+1})$.\\
        Формула: $y_{i+1} = y_{i} + h \cdot \frac{3}{2} \cdot \sqrt{x_{i+1}}$.\\
        
		\noindent
        \includegraphics[scale=0.45]{implicitEulerGraph_Step=0_5}
        \includegraphics[scale=0.45]{implicitEulerGraph_Step=0_25}\\
        \includegraphics[scale=0.45]{implicitEulerGraph_Step=0_125}
        \includegraphics[scale=0.45]{implicitEulerGraph_Step=0_05}
        
    \newpage
    \section*{Решение явным двушаговым методом Адамса}
        Выведем формулу:\\
        Интегральное представление точного решения:
        $y_{i+1} = y_{i} + \int_{x_{i}}^{x{i+1}} \! L_{k-1}(s) \, \mathrm{d}s$.\\
        Запишем многочлен Лагранжа $L_{k-1}(x) = [$при $k = 2] = L_{1}(x)$
        в форме Ньютона:\\
        $L_{1}(x) = f_{i} + f(x_{i}, x_{i-1})(x - x_{i}) = f_{i} +
        \frac{f_{i} - f{i-1}}{h}(x - x_{i})$.\\
        Подставим в интегральное представление. После интегрирования и
        приведения подобных получим формулу:
        $y_{i+1} = y_{i} + \frac{h}{2} (3f_{i} - f_{i-1}) = y_{i} +
        \frac{h}{2}(\frac{9}{2}\sqrt{x_{i}} - \frac{3}{2}\sqrt{x_{i-1}})$,
        где $i = 1, 2, .., n-1$.\\
        Будем проводить разгон методом Рунге-Кутты третьего порядка.\\
        $y_{1} = y_{0} + \frac{1}{6} k_{1} + \frac{1}{6} k_{2} + \frac{4}{6} k_{3}$.
        Где:\\
        $k_{1} = hf(x_{0}, y_{0}) = h \cdot \frac{3}{2} \sqrt{x_{0}} = 0$\\
        $k_{2} = hf(x_{0} + h, y_{0} + k_{1}) = hf(h, y_{0}) = h \cdot \frac{3}{2} \sqrt{h}$\\
        $k_{3} = hf(x_{0} + \frac{h}{2}, y_{0} + \frac{1}{4}k_{1} + \frac{1}{4}k_{2}) =
        hf(\frac{h}{2}, -1 + \frac{1}{4}k_{2}) = h \cdot \frac{3}{2} \sqrt{\frac{h}{2}}$\\
        То есть: $y_{1} = -1 + \frac{3h}{12}(\sqrt{h} + 4\sqrt{\frac{h}{2}}) =
        \frac{3h}{12}\sqrt{h}(1 + 4\sqrt{0.5}) - 1$.\\
        \includegraphics[scale=0.45]{adamsGraph_Step=0_5}
        \includegraphics[scale=0.45]{adamsGraph_Step=0_25}\\
        \includegraphics[scale=0.45]{adamsGraph_Step=0_125}
        \includegraphics[scale=0.45]{adamsGraph_Step=0_05}
        
    \section*{Вывод}
        На данном примере методы Эйлера почти полностью повторяют друг друга,
        в то время как метод Адамса сходится быстрее, поскольку он является
        многошаговым, то есть при пересчете следующих значений учитываются предыдущие
        шаги.
    \newpage
    
    \section*{Приложение}
        \begin{lstlisting}
def explicit_euler(y0, grid):
    grid_size = len(grid)
    assert (grid_size >= 2)
    h = grid[1] - grid[0]
    result = [(grid[0], y0)]
    for i in range(grid_size - 1):
        (xi, yi) = result[i]
        yi1 = yi + h * 1.5 * math.sqrt(xi)
        result.append((grid[i + 1], yi1))
    return result


def implicit_euler(y0, grid):
    grid_size = len(grid)
    assert (grid_size >= 2)
    h = grid[1] - grid[0]
    result = [(grid[0], y0)]
    for i in range(grid_size - 1):
        yi = result[i][1]
        xi1 = grid[i + 1]
        yi1 = yi + h * 1.5 * math.sqrt(xi1)
        result.append((xi1, yi1))
    return result


def two_step_adams(y0, grid):
    y1 = (3 * H / 12) * math.sqrt(H) * (1 + 4 * math.sqrt(0.5)) - 1
    result = [(grid[0], y0), (grid[1], y1)]
    for i in range(2, len(grid)):
        y_next = result[i - 1][1] + (H / 2) * 
	        (math.sqrt(grid[i - 1]) * 9.0 / 2.0 -
         	 math.sqrt(grid[i - 2]) * 3.0 / 2.0)
        result.append((grid[i], y_next))
    return result
        \end{lstlisting}
        
        
\end{document}
