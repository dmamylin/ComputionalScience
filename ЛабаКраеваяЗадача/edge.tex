\documentclass[12pt,a4paper]{article}
\usepackage[utf8]{inputenc}
\usepackage[russian]{babel}
\usepackage{listings}
\usepackage{amsmath}
\usepackage{geometry}
\usepackage{graphicx}
 \geometry{
 a4paper,
 total={210mm,297mm},
 left=5mm,
 right=5mm,
 top=5mm,
 bottom=5mm,
 }

%\topmargin=-1.5cm
%\parindent=24pt
%\parskip=0pt
%\flushbottom
%\lstset{language=python}

\title{Расчетно-графическая работа №3}
\author{Мамылин Дмитрий, МТ-301}

\begin{document}

\DeclareGraphicsExtensions{.png}

\maketitle
\thispagestyle{empty}

\newpage
\thispagestyle{empty}

    \section*{Постановка задачи}
        Дана краевая задача на отрезке $[0, 1]$:\\
        \[
        (1):
        \begin{cases}
        	y'' = y + 16.8 + 7.4 \cdot x(1 - x)\\
	    	y(0) = 0\\
     		y'(1) + y(1) = 2e + 1.7
     	\end{cases}
     	\]\\
     	где $f(x, y, y') = y + 16.8 + 7.4 \cdot x(1 - x)$.\\\\
     	
     	\noindent
        Необходимо решить ее методом стрельбы и методом прогонки, используя:\\
        \vspace{-5mm}
        \begin{itemize}
        \setlength\itemsep{-1.5em}
	        \item Метод Рунге-Кутта четвертого порядка для решения задачи Коши в методе стрельбы\\
    	    \item Метод Ньютона для решения нелинейного уравнения в методе стрельбы\\
        	\item Метод введения фиктивного узла для аппроксимации краевых условий в методе прогонки.
        \end{itemize}
        
    \section*{Точное решение}
        Найдем точное решение задачи $(1)$.\\
        Соответствующее однородное уравнение:\\
        $y'' - y = 0$\\
        Его характеристическое уравнение:\\
        $\lambda^{2} = 1 \Rightarrow \lambda_{1} = -1, \lambda_{2} = 1$\\
        Общее решение однородного: $y(x) = C_{1}e^{-x} + C_{2}e^{x}$\\
        Частное решение неоднородного: $y(x) = 7.4x^{2} - 7.4x - 2$\\
        Тогда общее решение неоднородного: $y(x) = C_{1}e^{-x} + C_{2}e^{x} +
        7.4x^{2} - 7.4x - 2$\\\\
        
        \noindent
        Найдем константы:\\
        $\begin{cases}
	        C_{1} + C_{2} = 2\\
    	    -C_{1}e^{-1} + C_{2}e + 7.4 = -C_{1}e^{-1} - C_{2}e + 2 + 2e + 1.7
        \end{cases}$\\
        
        \vspace{2mm}
        \noindent
        Откуда:\\
        $C_{2} = \frac{2e - 3.7}{2e}$\\
        $C_{1} = 2 - C_{2} = \frac{2e + 3.7}{2e}$\\\\
        
        \noindent
        Точное решение:\\
        $y(x) = \frac{2e + 3.7}{2e}e^{-x} + \frac{2e - 3.7}{2e}e^{x} 
        + 7.4x^{2} - 7.4x - 2$
    
    \newpage
    \section*{Метод стрельбы}
        Рассматриваем вспомогательную задачу Коши с некоторым параметром $\mu$:\\
        \[
        (2):
        \begin{cases}
        	y'' = y + 16.8 + 7.4 \cdot x(1 - x)\\
	        y(0) = 0\\
    	    y'(0) = \mu\\
        \end{cases}
        \]
        Сделаем замену: $z = y'$, тогда система $(2)$ примет вид:\\
        \[
        (2'):
        \begin{cases}
        	y' = z\\
        	z' = y + 16.8 + 7.4 \cdot x(1 - x)\\
        	y(0) = 0\\
        	z(0) = \mu\\
        \end{cases}
        \]
        Где:\\
        $f_{1}(x, y, z) = z$\\
        $f_{2}(x, y, z) = f(x, y, z) = f(x, y, y') = y + 16.8 + 7.4 \cdot x(1 - x)$.\\\\
        
        \noindent
        Общий вид метода Рунге-Кутта четвертого порядка для системы ОДУ второго порядка:\\
        $\begin{cases}
        	y' = f_{1}(x, y, z)
        	z' = f_{2}(x, y, z)
        	y(x_{0}) = y_{1,0}
        	z(x_{0}) = y_{2,0}
        end{cases}$\\\\
        
        \noindent
        $y_{i+1} = y_{i} + \frac{1}{6}(k_{i}^{1} +2k_{i}^{2} + 2k_{i}^3 + k_{i}^4)$\\
        $z_{i+1} = z_{i} + \frac{1}{6}(l_{i}^{1} +2l_{i}^{2} + 2l_{i}^3 + l_{i}^4)$\\
        
        \noindent
        Где:\\
        $k_{i}^{1} = hf_{1}(x_{i}, y_{i}, z_{i})$\\
        $l_{i}^{1} = hf_{2}(x_{i}, y_{i}, z_{i})$\\
        $k_{i}^{2} = hf_{1}(x_{i} + \frac{h}{2}, y_{i} + \frac{k_{i}^{1}}{2},
            z_{i} + \frac{l_{i}^{1}}{2})$\\
        $l_{i}^{2} = hf_{2}(x_{i} + \frac{h}{2}, y_{i} + \frac{k_{i}^{1}}{2},
            z_{i} + \frac{l_{i}^{1}}{2})$\\
        $k_{i}^{3} = hf_{1}(x_{i} + \frac{h}{2}, y_{i} + \frac{k_{i}^{2}}{2},
            z_{i} + \frac{l_{i}^{2}}{2})$\\
        $l_{i}^{3} = hf_{2}(x_{i} + \frac{h}{2}, y_{i} + \frac{k_{i}^{2}}{2},
            z_{i} + \frac{l_{i}^{2}}{2})$\\
        $k_{i}^{4} = hf_{1}(x_{i+1}, y_{i} + k_{i}^{3}, z_{i} + l_{i}^{3})$\\
        $l_{i}^{4} = hf_{2}(x_{i+1}, y_{i} + k_{i}^{3}, z_{i} + l_{i}^{3})$\\\\
        
        \noindent
        Применяем к данной задаче:\\
        $y_{i+1} = y_{i} + \frac{1}{6}(k_{i}^{1} +2k_{i}^{2} + 2k_{i}^3 + k_{i}^4)$\\
        $z_{i+1} = z_{i} + \frac{1}{6}(l_{i}^{1} +2l_{i}^{2} + 2l_{i}^3 + l_{i}^4)$\\
        Где:\\
        $k_{i}^{1} = hz_{i}$\\
        $l_{i}^{1} = h(y_{i} + 7.4x_{i}(1 - x_{i}) + 16.8)$\\
        $k_{i}^{2} = h(z_{i} + \frac{l_{i}^{1}}{2})$\\
        $l_{i}^{2} = h(y_{i} + \frac{k_{i}^{1}}{2} + 7.4(x_{i} + \frac{h}{2})
            (1 - \frac{h}{2} - x_{i}) + 16.8)$\\
        $k_{i}^{3} = h(z_{i} + \frac{l_{i}^{2}}{2})$\\
        $l_{i}^{3} = h(y_{i} + \frac{k_{i}^{2}}{2} + 7.4(x_{i} + \frac{h}{2})
            (1 - \frac{h}{2} - x_{i}) + 16.8)$\\
        $k_{i}^{4} = h(z_{i} + l_{i}^{3})$\\
        $l_{i}^{4} = h(y_{i} + k_{i}^{3} + 7.4x_{i+1}(1 - x_{i+1}) + 16.8)$\\\\
        
        \noindent
        Ищем параметр $\mu$. Решаем уравнение: $F(\mu) = 0$.\\
        Где $F(\mu) = z(\mu, 1) + y(\mu, 1) - 2e - 1.7$\\
        Применяем метод Ньютона:\\
        $\mu_{j+1} = \mu_{j} - F(\mu_{j}) / F'(\mu_{j})$,
        где $\mu_{0} = 1$.\\\\
        
		\noindent
		Для поиска $F'(\mu_{j}) = y'_{\mu}(\mu, 1)$ продифференцируем задачу Коши по параметру $\mu$:\\
		$y''_{x\mu} = z'_{\mu}$\\
		$z''_{x\mu} = y'_{\mu} f'_{y}(x, y, z) + z'_{\mu} f'_{z}(x, y, z)$\\
		$y'_{\mu}(0) = 0$\\
		$z'_{\mu}(0) = 1$\\
		Обозначим: $v = y'_{\mu}; u = z'_{\mu}$\\\\
		
		\noindent
		Получим новую задачу Коши:\\
		$v' = u = f_{1}(x, v, u)$\\
		$u' = v = f_{2}(x, v, u)$\\
		$v(0) = 0$\\
		$u(0) = 1$\\\\
		
		\noindent
		Для ее решения применяем метод Рунге-Кутта:\\
		$v_{i+1} = v_{i} + \frac{1}{6}(s_{i}^{1} +2s_{i}^{2} + 2s_{i}^3 + s_{i}^4)$\\
        $u_{i+1} = u_{i} + \frac{1}{6}(t_{i}^{1} +2t_{i}^{2} + 2t_{i}^3 + t_{i}^4)$\\
        
        \newpage
        \noindent
        Где:\\
        $s_{i}^{1} = hu_{i}$\\
        $t_{i}^{1} = hv_{i}$\\
        $s_{i}^{2} = h(u_{i} + \frac{s_{i}^{1}}{2})$\\
        $t_{i}^{2} = h(v_{i} + \frac{t_{i}^{1}}{2})$\\
        $s_{i}^{3} = h(u_{i} + \frac{s_{i}^{2}}{2})$\\
        $t_{i}^{3} = h(v_{i} + \frac{t_{i}^{2}}{2})$\\
        $s_{i}^{4} = h(u_{i} + s_{i}^{3})$\\
        $t_{i}^{4} = h(v_{i} + t_{i}^{3})$\\\\
        
        \noindent
        Найдем $F'(\mu_{0}) = F'(1)$
\end{document}
